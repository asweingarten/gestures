\documentclass{article}

\usepackage{fullpage}
\usepackage{url}
\usepackage{verbatim}
\usepackage{graphicx}
\usepackage{parskip}
\usepackage{pdfpages}

\begin{document}

\section{Test}\label{test}

\section{Abstract}

\section{Problem Statement and Solution Motivation}
Wearable technology, such as smart watches, are poised to become a ubiquitous consumer product. Many user interface paradigms still need to be rethought for these devices to accommodate their small screens and limited input methods.
Voice commands may be part of the solution, but there are many situations where they are ineffective, due to privacy, politeness, or excessive background noise.
Hand and arm gesture recognition is another ideal input method for smart watches. Though less expressive than voice commands, these movements are relatively private and they are a natural way to communicate with the environment.
There are presently no publicly available gesture recognition solutions for off-the-shelf smart watches, though theories and interface ideas have been prototyped with other sensors such as Microsoft Kinect or custom ultra-sound detection systems.
The potential applications of this technology are far-reaching, from seamless control of Internet-of-things devices to accessibility for visually impaired users.
We have acquired a LG G Watch and a Nexus 5 phone to record gesture data via a simple, custom-built Android  application. Data collected from the watch’s accelerometer, magnetometer, and gyroscope will be used to train a neural network to classify a set of simple gestures. We will compare the accuracy and performance of our classifier against similar gesture recognition systems in the research community.

\section{Theoretical Background Material}

\end{document}
